\section{Naturaleza de la economía matemática}

\subsection{Economía matemática vs economía no matemática}

Tanto la economía matemática como la no matemática no son excluyentes una de otra, sin embargo, la economía
matemática presenta ventajes en tanto a su precisión y análisis, ya que no solo se basa en el supuesto teórico,
sino también en la comprobación y la certeza de afirmación.

\begin{itemize}
    Las ventajas de la economía matemática son:
    \item El lenguaje usado es más conciso y preciso.
    \item Hay una gran cantidad de teoremas disponibles.
    \item Como obliga a expresar de forma explícita la suposición, se evita la adopción 
    no intencional de suposiciones implicitas indeseables.
    \item Permite tratar el caso general de \textit{n} variables.
\end{itemize}

\subsection{Economía matemática vs econometria}

La econometria es la medición de datos económicos a traves de metodos estadísticos de estimación y prueba de hipótesis.
En cambio, la economía matemática es la aplicación matemática a los aspectos puramente teóricos del analisis económico.
