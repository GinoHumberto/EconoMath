\section{Modelos económicos}

Un modelo económico es una abstracción del mundo real. Consiste en un conjunto de ecuaciones que representan, de manera simplificada, 
las relaciones entre variables con el fin de describir y analizar la estructura y el comportamiento de un fenómeno, en este caso económico.

\subsection{Elementos de un modelo matemático}

\subsubsection{Variables, constantes y parámetros}

Una \textit{variable} es algo cuya magnitud puede cambiar. En economía, las variables más utilizadas suelen ser el
precio, ganancia, costo, ingreso nacional, consumo, inversión, importaciones y exportaciones.

Hay dos tipos de variables:
\begin{itemize}
    \item Endogenas: aquellas cuyos \textit{valores de solución} se originan desde dentro, desde el mismo modelo.
    Por ejemplo el nivel de producción maximización-ganancia.
    \item Exógenas: aquellas determinadas por fuerzas externas al modelo
\end{itemize}

Una \textit{constante} es una magnitud que \textbf{no} cambia. 

Una \textit{constante parametrica o parametro} es una constante a la cual se le puede asignar un valor, es decir 
es una constante que es variable.

\subsubsection{Ecuaciones e identidades}

En la economía hay 3 tipos de ecuaciones:
\begin{itemize}
    \item Ecuaciones definicionales: establece una identidad\footnote{Identidad es una definición, una verdad absoluta. Por ejemplo \(1 \equiv 1\)} 
    entre 2 expresiones que tienen \textit{el mismo significado}. Para una ecuación definicional se utiliza el simbolo \equiv.
    Por ejemplo:
    \[
    \pi \equiv R - C
    \]
    Donde \(R\) es el exeso de ingreso total, \(C\) es el costo total y \(\pi\) es el la ganancia total 
    \item Ecuaciones de comportamiento: especifíca como se comporta una variable en respuesta a cambios en otras variables.Por ejemplo 
    cuanto gasta la gente en función de sus ingresos.
    \[
    C = \alpha + bY
    \]
    Donde \(C\) es el consumo total, \alpha el consumo autónomo(Lo que la gente gasta incluso si no tiene ingresos), \(b\) la propersión a consumir, \(Y\) Ingreso nacional
    \item Ecuaciones condicionales: se debe satisfacer un requerimiento. Por ejemplo la condición de equilibro como requisito en el caso de:
    \begin{gather*}
        Q_d = Q_s \quad \text{Cantidad demandada} = \text{Cantidad suministrada} \\ S = I \quad \text{Ahorro previsto} = \text{inversión prevista}
    \end{gather*}
\end{itemize} 