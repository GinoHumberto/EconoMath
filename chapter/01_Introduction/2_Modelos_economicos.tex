\section{Modelos económicos}

Cualquier teoría económica no es más que una abstracción del mundo real. Entonces un modelo económico
es una simplificación de la realidad, que toma unas variables para modelar un sistema.

\subsection{Elementos de un modelo matemático}

Cuando un modelo es matemático consiste de un conjunto de ecuaciones para describir la estructura del modelo.

\subsubsection{Variables, constantes y parámetros}

Una \textit{variable} es algo cuya magnitud puede cambiar. En economía, las variables más utilizadas suelen ser el
precio, ganancia, costo, ingreso nacional, consumo, inversion, importaciones y exportaciones.

Hay dos tipos de variables:
\begin{itemize}
    \item Endogenas: son aquellas cuyos \textit{valores de solución} se originan desde dentro, desde el mismo modelo.
    Por ejemplo el nivel de producción maximización-ganancia.
    \item Exógenas: son aquellas determinadas por fuerzas externas al modelo
\end{itemize}

Una \textit{constante} es una magnitud que no cambia. 

Una \textit{constante parametrica o parametro} es una constante a la cual se le puede asignar un valor, es decir 
es una constante que es variable.

\subsubsection{Ecuaciones e identidades}